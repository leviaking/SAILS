\documentclass[11pt]{article}
\usepackage[margin=1.4in]{geometry}                % See geometry.pdf to learn the layout options. There are lots.
\geometry{letterpaper}                   % ... or a4paper or a5paper or ... 
%\geometry{landscape}                % Activate for for rotated page geometry
%\usepackage[parfill]{parskip}    % Activate to begin paragraphs with an empty line rather than an indent
\usepackage{parskip}
%\usepackage{graphicx}
%\usepackage{amssymb}
%\usepackage{epstopdf}
%%
%\usepackage{rotating}
%\usepackage{tikz}
%\usetikzlibrary{shapes,arrows}
%\usepackage{graphicx}
%\usepackage{tikz-dependency}
\usepackage{natbib}
\usepackage{url}
\usepackage{color,soul}
%
%\DeclareGraphicsRule{.tif}{png}{.png}{`convert #1 `dirname #1`/`basename #1 .tif`.png}
\setlength{\parindent}{15pt} %Paragraphs will not be indented
%\parskip=10pt
\linespread{1.2}

\title{Semantic Analysis of Learner Sentences}
\author{Levi King \\ Dissertation Proposal}
%\date{April 15, 2015}

\begin{document}
\maketitle

\section{Timeline}
%\begin{table}[h!] %tb!]
\begin{center}
\begin{tabular}{|l|l|l|}
  \hline
  \textbf{Task} & \textbf{Start} & \textbf{Finish} \\
  \hline
  \hline
  Finalize committee & & February 2016 \\
  \hline
  \hline
  System development & January 2016 &  \\
  \hline
  Data collection materials, IRB forms & July 2016 & October 2016 \\
  \hline
  Collect data & October 2016 & November 2016 \\
  \hline
  Design annotation scheme (features) & February 2017 & May 2017 \\
  \hline
  Annotate data (features) & June 2017 & September 2017 \\
  \hline
  \textit{Write introduction} & September 2017 & October 2017 \\
  \hline
  Design annotation scheme (holistic) & September 2017 & October 2017 \\
  \hline
  Annotate data (holistic) & September 2017 & October 2017 \\
  \hline
  System experiments with new data & October 2017 & February 2018 \\
  \hline
  \textit{Write SLA section (motivation, lit review)} & October 2017 & November 2017 \\
  \hline
  \textit{Write data collection \& description section} & October 2017 & November 2017 \\
  \hline
  \textit{Write (NLP) literature review} & November 2017 & December 2017 \\
  \hline
  \textit{Write method section (including training data)} & December 2017 & January 2017 \\
  \hline
  \textit{Write experiments \& results section} & January 2017 & February 2017 \\
%  \hline
%  Implement feedback system & December 2016 & January 2017 \\
%  \hline
%  \textit{Write feedback section} & December 2016 & February 2017 \\
  \hline
  \textit{Write conclusion (finish 1st draft)} & February 2017 & March 2017 \\
%  \hline
%  Complete any follow-up experiments & February 2017 & March 2017 \\
  \hline
  \textit{Revisions} & April 2018 & April 2018 \\
  \hline
  \textit{Final draft complete} / Defense & & May 2018 \\
  \hline
\end{tabular}
\end{center}
%\end{table}

%\section{Leftovers}
%\begin{enumerate}
%\item{We've been using the triple-based error annotation from our semantic triples experiments, but that was mostly a matter of convenience. What kind of annotation do we really want now? \textbf{What annotation is essential for us to be able to evaluate our system?}}
%\item{Data collection will be one of the first tasks we encounter. Should we (can we) stick with single (non-sequential) PDT items for simplicity's sake? If we use sequential PDTs, we'll need to think about anaphora resolution and maybe NER, etc.}
%\item{How much data will we need? (How many items? How many NNS participants? How many NS participants (and how many responses per item from each NS)? Should we hold out some data for testing?}
%\item{At some point in this process, we should publish our data sets (including the old one). Where should we fit this in the timeline, and how can we ensure that we submit to the IRB and collect our data in such a way to minimize the complications of publishing the data? (Mostly a ``note to self''...)}
%\item{I really like the feedback idea: Present the user with the most similar response in the gold standard and give some minimal information about how this GS sentence relates to the rest of the GS. I think it's a clever use of what we will have at hand (``good'' responses from the GS). I also think it \textit{seems} useful, but I'd like to see if I can find a solid SLA justification for it. From a practical consideration, I think it should be relatively simple to implement. I also think it fits fairly well with our intention to keep this project somewhat modular; it might not be the ideal form of feedback, but I think it seems like a nice ``stopping point'' for us on that front, and it would be a nice starting point for anyone wanting to expand our system to include more thorough feedback. How exactly will we determine which GS sentence is most similar to the NNS sentence? What kind of information about the sentence and the GS will we provide, and how will we derive that information?}
%\item{There are ideas that have come up repeatedly that I have not explicitly added to the timeline, namely semantic role labeling and WordNet. What can we say about these things now? Beyond simply allotting extra time for the experiments, do we have thoughts on how and where to incorporate these ideas?}
%\item{MD: How are PDTs used in second language assessment? Is a GS used? Or are all responses hand graded? (Check Chodorow, etc.)}
%\item{What does SLA literature have to say about ``nativeness''? Who works on this? What can we find?}
%\item{MD: Start forming PDT stuff; wrap up clustering, write up clustering -- keep in mind data is limited (mostly only transitives), so we don't know how well it would generalize; start lit review.}
%\item{MD: possible question: how lightweight can it be / does it need to be?}
%\end{enumerate}

\bibliographystyle{plainnat}
\bibliography{levi-bib}

\end{document}

%Start putting together BEA paper based on the work from this semester (clustering, etc.). Identify any gaps in what we need. (BEA deadline early March).

%%from 1/13 meeting:
%%research questions for proposal
%%writing diss intro and/or BEA paper
%%recover the BEA paper status (what remains to be done, etc.)
%%new data collection materials; what kind of things do we want? what are the criteria for new materials?

%% Next meeting, Tuesday, 2/2, 11am.

%%%motivation, looking toward BEA: various sentence types, there's more than yes or no to the quality of a response. Our scoring (vs. strict matching) also lends itself better to interesting feedback.
%%% We know the GS will be limited; we want to optimize it, but still, we need a more robust way of judging (SCORING) responses.

%%%BEA paper, prospectus.


%%%Remove lots of "we" language; can use "I", but try to avoid.



